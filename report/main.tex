\documentclass[12pt,a4paper,notitlepage]{report}
\usepackage[utf8]{inputenc}
\usepackage[english]{babel}

% AMS packages
\usepackage{amsmath}
\usepackage{amsfonts}
\usepackage{amssymb}
\usepackage{graphicx}

\usepackage{eurosym}% euros
\usepackage{listings} % code

\usepackage{todonotes}
\title{LINMA 2470 : Stochastic models}
\author{Quentin Laurent}
\begin{document}
\maketitle
\section*{Model}
We are going to model our bank as a $M/G/2$ queue. Hence the arrival times will follow an exponential law, and the distribution of service time is any distribution where times are positive.
\todo[inline]{Comment traiter le fait qu'un teller est parfois inactif?}
\section*{Parameters estimation}
In order to determine the arrival rate $\lambda (hr^{-1})$ of the clients in the bank we need to analyse the provided data.\\
If we assume that the number of people in the line at the end of the half-day is zero, then the number of people who arrived is equal to the number of people who were served by either one of the two tellers.\\
A more sophisticated approach would include time-varying arrival rate.
\todo[inline]{Traiter les données et montrer pourquoi c'est le meilleur estimateur}


\section*{Anylogic model}

\todo[inline]{Modèle anylogic}

\section*{The merge project}
 Since the clients arriving in the two old branches will sum up and go to the merged branch, we simply add the two Poisson processes, which gives us a new Poisson process with an arrival rate of the clients twice as big as that of the two old branches.

But how many tellers do we need? Since in each bank a teller is idle at some time, how many of them is needed to make sure that the queue doesn't grow too much?
\todo[inline]{answer + validation with anylogic}


\end{document}