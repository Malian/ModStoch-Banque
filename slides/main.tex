\documentclass[10pt]{beamer}

\usepackage[english]{babel}
\usepackage[utf8]{inputenc}
\usepackage[T1]{fontenc}
\usepackage{lmodern}

\usepackage{layout}
\usepackage{epsfig}
\usepackage{graphicx}
\graphicspath{{../report/images/}}
\usepackage{subfigure}


\usepackage{siunitx}
\usepackage{eurosym}

\usepackage{amsthm}
\usepackage{amsmath}
\usepackage{amssymb}

\usepackage{mathrsfs}
\usepackage{wrapfig}
\usepackage{url}
\usepackage{multirow}
\usepackage{array}
\usepackage{pgfplots}

\usepackage[version=3]{mhchem}

\usepackage{wasysym}
%Bibtex
%\usepackage[square]{natbib}
%\newcommand{\newblock}{}

\usetheme{Warsaw}
\setbeamertemplate{headline}{}

\usepackage{graphicx}
\usepackage{epsfig}
\usepackage{epstopdf}

\usepackage{todonotes}
\title{Bank queue optimization}
\author{
  Malian De Ron
  \and
  Florentin Goyens
  \and
  Quentin Laurent
  \and
  Harold Taeter
}

\usepackage{tikz}
\tikzstyle{vertex}=[circle,fill=gray!50,minimum size=15pt,inner sep=0pt]
\tikzstyle{visited}=[circle,fill=green!25,minimum size=15pt,inner sep=0pt]
\tikzstyle{unvisited}=[circle,fill=blue!25,minimum size=15pt,inner sep=0pt]



\begin{document}

\begin{frame}
  \maketitle
\end{frame}

\begin{frame}
  \frametitle{Model}
  \begin{block}{$M/G/2$ model}
  \begin{itemize}
    \item Time between two arrivals follows an exponential law
    \item 2 tellers
    \end{itemize}
  \end{block}
\end{frame}


\begin{frame}
  \frametitle{Parameters estimation}
 
  
\end{frame}

\begin{frame}
  \frametitle{Anylogic model}
  
\end{frame}

\begin{frame}
  \frametitle{The merge project}
  \begin{block}{Modification of the model}
  \begin{itemize}
  \item  Since the clients arriving in the two old branches will sum up and go to the merged branch, we simply add the two Poisson processes, which gives us a new Poisson process with an arrival rate of the clients twice as big as that of the two old branches.
  \item Processing time follows the same distribution as before
  \end{itemize}
  \end{block}
  
  \begin{block}{Estimating the number of tellers needed}
  Fluid approximation
  \end{block}
\end{frame}

\begin{frame}
\frametitle{Optimum number of servers}
\begin{block}{What is our cost?}
\begin{itemize}
\item Proportional to the capacity we can handle and the waiting time of the clients
\begin{eqnarray*}
\min_c & & \alpha c + \beta \int_D A(t)-S(t)\\
\end{eqnarray*}
where $D = \{t | A(t)>=S(t) \}$
\item The number of tellers must be an integer $\rightarrow$ additional constraint : $N_t = c \cdot \mu$ must be integer
\end{itemize}
\end{block}
\begin{figure}
\centering
\begin{tabular}{|c|c|c|c|}
\hline 
1/2 day salary(\euro ) & Half day time(s) & $\alpha$(\euro $s$/clients) & $\beta$(\euro/clients$\cdot s$) \\ 
 \hline 
$75$ & $10800$ & $0.433526$ & $3.47222e-07$ \\ 
 \hline 
 
 \end{tabular}
\caption{Parameters\label{table:param}}
\end{figure}
\end{frame}

\begin{frame}
\frametitle{Results : who gets fired?}
\begin{figure}
\centering
\subfigure[lundi pm]{
\includegraphics[width = 5cm]{half\string_daylundi\string_pm\string_clients\string_served.eps}
}
\subfigure[vendredi am]{
\includegraphics[width = 5cm]{half\string_dayvendredi\string_am\string_clients\string_served.eps}
}
\end{figure}
\begin{center}
$N_t = $ \input{../report/tex_matlab/tellers.tex}
\end{center}
\end{frame}
\end{document}
