\documentclass[10pt]{beamer}

\usepackage[english]{babel}
\usepackage[utf8]{inputenc}
\usepackage[T1]{fontenc}
\usepackage{lmodern}

\usepackage{layout}
\usepackage{epsfig}
\usepackage{graphicx}
\graphicspath{{images/}}

\usepackage{siunitx}

\usepackage{amsthm}
\usepackage{amsmath}
\usepackage{amssymb}

\usepackage{mathrsfs}
\usepackage{wrapfig}
\usepackage{url}
\usepackage{multirow}
\usepackage{array}
\usepackage{pgfplots}

\usepackage[version=3]{mhchem}

\usepackage{wasysym}
%Bibtex
%\usepackage[square]{natbib}
%\newcommand{\newblock}{}

\usetheme{Warsaw}
\setbeamertemplate{headline}{}

\usepackage{graphicx}
\usepackage{epsfig}
\usepackage{epstopdf}

\usepackage{todonotes}
\title{Bank queue optimization}
\author{
  Malian De Ron
  \and
  Florentin Goyens
  \and
  Quentin Laurent
  \and
  Harold Taeter
}

\usepackage{tikz}
\tikzstyle{vertex}=[circle,fill=gray!50,minimum size=15pt,inner sep=0pt]
\tikzstyle{visited}=[circle,fill=green!25,minimum size=15pt,inner sep=0pt]
\tikzstyle{unvisited}=[circle,fill=blue!25,minimum size=15pt,inner sep=0pt]



\begin{document}

\begin{frame}
  \maketitle
\end{frame}

\begin{frame}
  \frametitle{Model}
  \begin{block}{$M/G/2$ model}
  \begin{itemize}
    \item Time between two arrivals follows an exponential law
    \item 2 tellers
    \end{itemize}
  \end{block}
\end{frame}


\begin{frame}
  \frametitle{Parameters estimation}
 
  
\end{frame}

\begin{frame}
  \frametitle{Anylogic model}
  
\end{frame}

\begin{frame}
  \frametitle{The merge project}
  \begin{block}{Modification of the model}
  \begin{itemize}
  \item  Since the clients arriving in the two old branches will sum up and go to the merged branch, we simply add the two Poisson processes, which gives us a new Poisson process with an arrival rate of the clients twice as big as that of the two old branches.
  \item Processing time follows the same distribution as before
  \end{itemize}
  \end{block}
  
  \begin{block}{Estimating the number of tellers needed}
  Fluid approximation
  \end{block}
\end{frame}

\begin{frame}
\frametitle{Fluid approximation}


\end{frame}


\end{document}
